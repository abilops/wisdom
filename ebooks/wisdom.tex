% XeLaTeX can use any Mac OS X font. See the setromanfont command below.
% Input to XeLaTeX is full Unicode, so Unicode characters can be typed directly into the source.

% The next lines tell TeXShop to typeset with xelatex, and to open and save the source with Unicode encoding.

%!TEX TS-program = xelatex
%!TEX encoding = UTF-8 Unicode

\documentclass[12pt, twosided]{book}
\usepackage{geometry}                % See geometry.pdf to learn the layout options. There are lots.
\usepackage{fancyhdr}
\geometry{a5paper}                   % ... or a4paper or a5paper or ... 
%\geometry{landscape}                % Activate for for rotated page geometry
%\usepackage[parfill]{parskip}    % Activate to begin paragraphs with an empty line rather than an indent
\usepackage{graphicx}
\usepackage{hyperref}
\usepackage{amssymb}
\usepackage{pdfpages}

% Page numbering location
%\pagestyle{fancy}
%\fancyhf{} % clear all header and footer fields
%\fancyfoot[CO,CE]{\texttt{ page \thepage}} %RO=right odd, RE=right even
%\renewcommand{\headrulewidth}{0pt}
%\renewcommand{\footrulewidth}{0pt}

% Will Robertson's fontspec.sty can be used to simplify font choices.
% To experiment, open /Applications/Font Book to examine the fonts provided on Mac OS X,
% and change "Hoefler Text" to any of these choices.

\usepackage{fontspec,xltxtra,xunicode}
\defaultfontfeatures{Mapping=tex-text}

%%% GITHUB does not have good fonts, so sticking to defaults
%\setromanfont[Mapping=tex-text, Size=30]{Times New Roman}
%\setsansfont[Scale=MatchLowercase,Mapping=tex-text]{Gill Sans}
%\setmonofont[Scale=MatchLowercase]{Andale Mono}

\title{Wisdom Project}
\author{Merlin Mann}
%\date{}                                           % Activate to display a given date or no date


\begin{document}%
% Cover image!
\raggedright

% Formatting for epigraphs
\newcommand{\epi}[2]{
		\begin{quote}
				{#1}
		\end{quote}
		-- #2
\vspace{2em}
}

% For the Markdown "----"
\newcommand{\dashdash}{[3em]$\ast\ \ast\ \ast$ \\}

% Voi
\newcommand{\remark}[1]{\textbullet #1\\}

% Management
\newcommand{\byline}[1]{\hfill –#1}

% Counter for each line of wisdom
\newcounter{wcounter}
\setcounter{wcounter}{0}
% Vertically center by removing those comments
\newcommand{\middleset}[1]{%
%\vspace*{\fill}
#1 %
%\vspace*{\fill}
}

% Set the wisdom line
\newcommand{\wisdom}[2][]{%
		% Increment counter only if this is a top
		\ifstrequal{#1}{}{\stepcounter{wcounter}}{}
		% break page only if this is a top-level and not corollary or related
		\ifstrequal{#1}{}{\clearpage}{\\[2em]}
				\ifstrequal{#1}{}{
				\begin{center}
						\textbf{$\bullet$\thewcounter}
				\end{center}
				}{%
						\emph{\Large {#1:}}%
				}
		\middleset{\Large #2}
}

% turn off page numbers
\pagenumbering{gobble}
% cover image
\includepdf{cover.jpg}

\begin{center}
		{\Huge {Merlin's Wisdom Project}} \\[7em]
		{\Large or} \\[2em]
		{\LARGE \emph{``Everybody likes being given a glass of water.''}}
		\vfill
		{\small It's only advice for you because it \textbf{had} to be advice for me.}
		\\[2em]
		{\emph{\small\href{https://wisdom.limo}{wisdom.limo $\nearrow$}}}
\end{center}

\cleardoublepage
\chapter*{Epigraphs}
\epi{Yet here, Laertes! aboard, aboard, for shame! 
\\ The wind sits in the shoulder of your sail,
\\ And you are stay'd for.}
{\emph{Hamlet}, Act 1, Scene 3.}

\epi{We are what we pretend to be, so we must be careful about what we pretend to be.}
{Kurt Vonnegut, Jr., \emph{Mother Night} (1962)}

\epi{At all times keep your crap detector on. If I say something that helps, good. If what I say is of no help, let it go. Don’t start arguments. They are futile and take us away from our purpose.}
{Richard Hugo, \emph{The Triggering Town} (1979)}

% Reset page numbers
\setcounter{page}{1}

% Resume numbering
% \pagenumbering{arabic}

% Start on an odd page
\cleardoublepage

\section*{Voi ch'entrate…}
Brief introductory remarks regarding the Project:\\
\remark{The ideas on offer here are based on my own experiences and sensibilities. They are mostly things that, above all, I have needed to learn.}
\remark{As it happens, most of the ideas have been painful to learn and even more difficult to practice.}
\remark{Some of the ideas are, admittedly, just my opinion.}
\remark{Many of the ideas will seem to contradict one other. Unfortunately, that doesn't mean one of them has to be wrong.}
\remark{None of the ideas will be true for every person or for all times. In the event that you ever successfully identify ideas that are true for every person and for all times, you may wish to start a religion.}
\remark{Also, none of the ideas is offered with the intention of being unkind, exclusionary, hurtful, or existentially ugly. If any comes across that away, I sincerely apologize in advance.}
\remark{Related: for any idea that strikes you as irrelevant or dumb or wrong or antithetical to your own experiences and sensibilities, please consider that it may not be, as we say, *for* you. The reader is encouraged to ignore or reject any ideas that they find undesirable.}
\remark{But, do bear in mind that often it's the idea that finds *you* undesirable. The better ideas usually have pretty high standards.}
\remark{These are ideas that I have believed to be true for myself at the time of composition. They are not immutable truths about The Universe, and I am open to changing my mind about any of them at any time.}
\remark{Ideas appended with a “thanks" are wisdoms that I gratefully learned from that credited person—often many years ago. They are not “submissions.” Because, alas, I am not *Reader’s Digest*.}
\remark{If you believe that it is possible to grow without change, you are probably neither growing nor changing.}
\remark{Wisdom is not a headline.}
\remark{Bulleted lists are a useful way to collect items that are either unrelated or may not benefit from being puffed into actual fancy prose. <!-- HELLO Also, embedded comments are good for saying hello to curious geeks. Hello. Welcome! -->}
\remark{None of the ideas should be interpreted as actual advice of any kind or for any purpose, and, thus, the Project is provided *as-is*.}
\remark{You should *not* rely upon this Project or its constituent ideas for *any purpose* without seeking legal, medical, emotional, spiritual, and/or directed career counseling.}
\remark{No glass containers, coolers, or inflammable materials will be permitted. No motorcycles after 3pm.}
\remark{The Project is never done.}
\byline{*The Management*}
% With the definitions done, we can just pour in the wisdoms!
\include{wisdom.md.tex}

\end{document}
