% XeLaTeX can use any Mac OS X font. See the setromanfont command below.
% Input to XeLaTeX is full Unicode, so Unicode characters can be typed directly into the source.

% The next lines tell TeXShop to typeset with xelatex, and to open and save the source with Unicode encoding.

%!TEX TS-program = xelatex
%!TEX encoding = UTF-8 Unicode

\documentclass[12pt, twosided]{book}
\usepackage{geometry}                % See geometry.pdf to learn the layout options. There are lots.
\geometry{a5paper}                   % ... or a4paper or a5paper or ... 
%\geometry{landscape}                % Activate for for rotated page geometry
%\usepackage[parfill]{parskip}    % Activate to begin paragraphs with an empty line rather than an indent
\usepackage{graphicx}
\usepackage{amssymb}
\usepackage{pdfpages}

% Will Robertson's fontspec.sty can be used to simplify font choices.
% To experiment, open /Applications/Font Book to examine the fonts provided on Mac OS X,
% and change "Hoefler Text" to any of these choices.

\usepackage{fontspec,xltxtra,xunicode}
\defaultfontfeatures{Mapping=tex-text}

%%% GITHUB does not have good fonts, so sticking to defaults
%\setromanfont[Mapping=tex-text, Size=30]{Times New Roman}
%\setsansfont[Scale=MatchLowercase,Mapping=tex-text]{Gill Sans}
%\setmonofont[Scale=MatchLowercase]{Andale Mono}

\title{Wisdom Project}
\author{Merlin Mann}
%\date{}                                           % Activate to display a given date or no date


\begin{document}%
% Cover image!
\includepdf{cover.jpg}
% Reset page numbers
\setcounter{page}{1}
% For the Markdown "----"
\newcommand{\dashdash}{\\[3em]$\ast\ \ast\ \ast$}
% Counter for each line of wisdom
\newcounter{wcounter}
\setcounter{wcounter}{0}
% Vertically center by removing those comments
\newcommand{\middleset}[1]{%
%\vspace*{\fill}
#1 %
%\vspace*{\fill}
}

% Set the wisdom line
\newcommand{\wisdom}[2][]{%
		% Increment counter only if this is a top
		\ifstrequal{#1}{}{\stepcounter{wcounter}}{}
		% break page only if this is a top-level and not corollary or related
		\ifstrequal{#1}{}{\clearpage}{\\[2em]}
				\ifstrequal{#1}{}{
				\begin{center}
						\textbf{$\bullet$\thewcounter}
				\end{center}
				}{%
						\emph{\Large {#1:}}%
				}
		\middleset{\Large #2}
}

% With the definitions done, we can just pour in the wisdoms!
\wisdom[]{Sometimes, an email is just a way to say, “I love you.”}
\wisdom[]{People think about you much less than you either hope or fear.}
\wisdom[]{It’s often easier not to be terrible.}
\wisdom[]{Whenever you’re not sure what to say, either say nothing, or ask a question.}
\wisdom[]{Be sparing in how often you tell someone their negative feelings are wrong; it rarely helps a sad person to be told that they are also a liar.}
\wisdom[Related]{feelings are *real*. (Thanks, Mike S.)}
\wisdom[]{Never organize anything you should discard.}
\wisdom[]{Organizing your email is like alphabetizing your recycling.}
\wisdom[]{As you cross the street, notice which car's driver feels most likely to do something stupid or dangerous. Walk a little slower, turn your head, and make direct eye contact. Brains cannot help but notice faces, plus eye contact startles anyone into suddenly remembering they live amongst other actual people.}
\wisdom[Related corollary]{navigate an urban sidewalk by *avoiding* eye contact. Not because you're anti-social, but because eyes tell you little about where your fellow pedestrians are headed. Monitor feet and footsteps for imminent direction; unfix your gaze ~two head-heights above the crowd to detect emerging patterns.}
\wisdom[]{If the thing you’re cooking doesn’t smell or sound like food yet, it’s probably not food yet.}
\wisdom[]{Buy slightly larger shoes.}
\wisdom[]{Say hello to every dog.}
\wisdom[]{Always wave back at children and all persons on trains.}
\wisdom[]{Flirt with all elderly women.}
\wisdom[]{Tip more.}
\wisdom[]{Look for the chimneys.}
\wisdom[]{[Keep moving *and* get out of the way](https://overcast.fm/+BmEPOwtek).}
\wisdom[]{*Never* give advice to a pregnant person unless they specifically asked for it.}
\wisdom[]{*Never* touch a pregnant person unless they specifically asked for it.}
\wisdom[]{*Never* tell a pregnant person horror stories about childbirth unless they specifically asked for it.}
\wisdom[Corollary]{When someone shares something scary about their birth story, it's probably because they're (justifiably) proud of surviving something that they never had imagined that they could survive.}
\wisdom[Related]{this also goes for most other scary personal anecdotes.}
\wisdom[]{Stop correcting people by immediately telling them what they "*should have said*." You are not helping.}
\wisdom[]{When you’re feeling awful and aren't sure what to do, pretend you are the person you love the most, and give *them* your best advice.}
\wisdom[]{Without active care and curation, any area in your home will tend to become "storage."}
\wisdom[Related]{"storage" is the least muscular or affirmative use of space in your life. Live and active areas represent future possibility; "storage" is an emotionally costly way of warehousing the past.}
\wisdom[]{If you see someone photographing a group, offer to take the photo for them so they can get in the picture. Please do not steal their camera.}
\wisdom[Related]{when you shoot a group photo, always take *at least* five shots from *at least* two angles. For the last couple photos, say: "Everybody say ***'BUTTS!'***" You will instantly get many totally natural smiles, plus you just gave them a fun story.}
\wisdom[]{In photography—as in life—always keep the light behind you.}
\wisdom[]{Sometimes, a person will confess something embarrassing that obviously makes them feel really dumb and vulnerable. That is *never* the time to say "I told you so," and it is rarely the best moment to offer advice that they never asked for. Just shut the fuck up and listen.}
\wisdom[]{If you have a small household responsibility—no matter how lame or quotidian—just do it now and without being asked. If you think the trash may need to go out, do not "check" to see if the trash needs to go out. Just take out the fucking trash. And quit reminding everybody you took the trash out. This is not Vietnam, and you are not a forgotten hero.}
\wisdom[Related]{the greatest curse of the middle-aged American man is the persistent belief that he is inadequately appreciated.}
\wisdom[]{Do not ask someone if they want a glass of water. Just bring them a glass of water. Everybody likes being given a glass of water.}
\wisdom[]{Buy the nicest screwdrivers you can afford.}
\wisdom[]{Every few months, take at least one panorama photo of your kid's room. At least annually, secretly record your kid talking for at least ten minutes. I promise you'll treasure both, and then you will curse yourself for not having done each way more often.}
\wisdom[]{Most well-written characters have something they want—or something they *think* they want. The more fascinating characters also have something they don’t want you to know. The best ones also have something they’re not pulling off nearly as well as they think.}
\wisdom[Related]{these are each also true for real people.}
\wisdom[]{Try always to store something in the first place you just looked for it. Not "where it's pretty" or "where we used to keep it" or "where we have more room." It goes *where it goes*—not where you **think** it goes.}
\wisdom[]{Almost every task in life benefits from the addition of a nearby trash bag.}
\wisdom[]{Just because you know something doesn't mean everybody knows it. Every day, somebody's born who's never seen *The Flintstones*.}
\wisdom[]{If an item is especially precious or valuable to you, never set it down anyplace that you wouldn't want it to be overnight.}
\wisdom[]{Call people what they'd like to be called. And, don't be a dick about it.}
\wisdom[]{Frequently ask yourself: do I want to be right, or do I want to be happy?}
\dashdash
\wisdom[]{Every project is a triangle made of time, money, and quality; shortening the length of one side necessarily lengthens one or—more often—*both* of the other sides.}
\wisdom[]{Less well known is that we each tend to blow it hardest in estimating the sides of the triangle we least understand or respect.}
\wisdom[]{Kindly note that the grave existential truth of the Project Management Triangle is *non-negotiable*. People **hate** this. Which is normal.}
\wisdom[]{If the person with whom you are negotiating finds it difficult to provide a decisive budget estimate for their project, ask them to try and situate it between two orders of magnitude. As in, "How many zeroes are we talking about here?" Quickly discovering that your ballpark figures are 2 to 6 zeroes apart can save you both a *lot* of time and frustration.}
\wisdom[Related]{if the client's estimate for any given aspect of the project feels poorly thought out, mentally double the estimated budget for money and time. At the end of all estimations, add *at least* another 20\% to the time and budget. You’re gonna need it, and, boy, are you ever going to earn it.}
\wisdom[Relatedly related]{have you leavened your estimate of the project with your hunches about the credibility of the client? Have you accounted for human foibles and flakiness in your estimate?}
\wisdom[]{When estimating the time it will take to do anything involving a child, add at least ten minutes per child. Make that 30 minutes for kids under five or over twelve.}
\wisdom[]{Always have a twelve-pack of Diet Coke and a good quantity of unflavored fizzy water. A lot of people love one (or both), and most of the people who do drink a lot of it.}
\wisdom[]{Whoever wants the meeting most usually holds the least power.}
\wisdom[]{Archive any email that’s older than 30 days. If it kills you to archive a given email, *immediately* turn it into a task, and then archive it.}
\wisdom[]{Most team culture comes out of a combination of what is tolerated and what is rewarded. If you legit want your culture to improve, change what you reward and rethink what you will tolerate.}
\wisdom[]{To better understand anyone's childhood, learn which things were *way* harder for them than they were for you.}
\wisdom[]{Any Slack is only as good as the guy who always has the latest post. (And, it's nearly always a guy.)}
\wisdom[]{Avoid any children’s movie whose theatrical trailer includes more than one fart or butt joke. That’s their idea of the best parts of the movie.}
\wisdom[]{If you don't remember what an app does, you can probably delete it.}
\wisdom[]{If you don't remember what a cable does, you can almost definitely discard it.}
\wisdom[]{The earlier a kid is around books often (and in *any* way), the earlier and easier their life of reading will go.}
\wisdom[]{Any time you locate a piece of digital information you were hunting for, tag it something like, "`\#OutboardBrain`." Chances are you'll want to find it again, and chances are you'll definitely forget it again.}
\wisdom[]{Avoid vegetarian dishes that aspire to approximate a recipe that’s typically based on meat.}
\wisdom[]{In any large retail store, choose the line that’s mostly young people who are by themselves.}
\wisdom[]{Always make ***all*** the bacon.}
\wisdom[]{Never try to bribe someone unless the amount you’re offering them feels *ludicrously* high.}
\wisdom[]{If you really want a glass of water at a restaurant, always order that first. As you do this, look the server in the eyes and nod.}
\wisdom[]{You’ll probably need to listen to at least three episodes of a podcast before you will know if you could *really* love it.}
\wisdom[]{If you want an honest opinion, ask for the second superlative. For example, if you want a thoughtful answer about someone's job, ask them their *second-least-favorite* thing about it.}
\wisdom[]{Avoid any food whose name has been altered for legal reasons. (Thanks, Chris C.)}
\dashdash
\wisdom[]{Sometimes, people ask you how you're doing when they're especially concerned about how *they’re* doing.}
\wisdom[]{Stay focused on the outcome, not your original strategy. Viz.: if you’re looking for *a* USB cable, don't fixate on finding a *specific* box that might contain a *specific* USB cable. Just find *a* goddamned cable.}
\wisdom[Related]{when you get stuck and frustrated about how to solve a problem, stop, take a breath, and ask yourself, “What am I *actually* trying to accomplish here?” Because, that’s the outcome on the other side of a new and less ambiguous strategy.}
\wisdom[Relatedly related]{whenever you catch yourself getting frustrated because you're hung up on some weird implementation detail, please say the following outloud in a firm but friendly tone: "*Just do the thing*."}
\wisdom[]{Before you freak out about how you are feeling right now, ask yourself how much (or how little) you're having of sleep, food, sex, water, exercise, alcohol, drugs, sunshine, human touch, family time, and probably some other stuff I don't know of but you definitely will.}
\wisdom[]{Whenever you need to carry two seemingly identical things (like, drinks or toothbrushes or what have you), always—*and only*—ever carry the one that’s yours in your right hand. When you pick up the two items, always mutter aloud to yourself, “*I’m always **right***.” Because, now, you are always right.}
\wisdom[]{After you’ve had two alcoholic beverages, begin alternating with equal amounts of water. If you have more than five drinks, change that ratio to two-to one in favor of water. (Thanks, Dennis. G.)}
\wisdom[]{Dinner parties and most large group meals are not really about eating. They’re mostly about easy socializing. So, if you get weird when you’re hungry, *eat before you arrive*. It’ll make everyone's evening more easy and more social.}
\dashdash
\wisdom[]{Sometimes in life, even though it's not your *fault*, it's still your problem. (Thanks, Marco A.)}
\wisdom[]{Being on time for things is a sign of character and respect. Adults who are pathologically late for things are unconsciously telling the world that other people's time is worthless to them.}
\wisdom[]{To clean out a junky drawer, remove the contents and put it in a box. Only when you've used one of the items in the box *twice* is it allowed to live in the drawer again. After a month, pitch or donate the remaining contents—or move the precious or useful stuff to deeper storage. But, yeah, you should probably just pitch it.}
\wisdom[]{Never argue on the internet. No one will remember whether you won or lost the argument; they'll just remember that you are the sort of person who argues on the internet.}
\wisdom[]{Whatever your problem is, remember that before you can get better, you have to stop getting worse. Try first to stop getting worse.}
\wisdom[]{Don't let people tell you whether, when, or how to season your food. It's your body.}
\wisdom[]{Just in general: never explain food. Yes, I see the provided sauce. And, no, I do not need a webinar on how it should be deployed.}
\wisdom[]{Whenever someone demands you change who you are, it's useful to ask yourself what they stand to gain from you agreeing to become someone else.}
\wisdom[]{To an anxious person, it often feels like the only way to achieve relaxed certainty is to keep seeking new information. But, remember that the more you know, the more you'll realize you don't know. And, then, you'll usually just find yourself fretting about getting more and more information, et cetera. Incline yourself towards getting out of the information centrifuge.}
\wisdom[]{Everybody grieves differently. You're not the mourning police, and no one benefits from you telling them they're being sad wrong.}
\wisdom[]{"Experience" is rarely the verb you're looking for. Reword your sentence with a more clear and muscular focus on what actually happened—and who or what caused it to happen. So, maybe don't say "I am experiencing technical difficulties" if you really mean "I broke the internet." You're not fooling anyone.}
\wisdom[Related]{Please don't say "impact" (v.) if you just mean "affect" (v.). It makes you sound like a lame PowerPoint about dentistry.}
\wisdom[]{Few journalists get to choose the headline for their piece. So, whenever the clickbait of a terrible, search-engine-optimized headline belies an actually-good article, consider getting mad at the editor. Not the writer.}
\dashdash
\wisdom[]{You are not obligated to have a strong opinion about everything. Get fewer opinions about way fewer things, and then strive always to interrogate the basis of your strongest opinions. This is very difficult, so be grateful if you've found fewer strong opinions to interrogate.}
\wisdom[]{Priorities are like arms. If you think you have more than a couple, you're either lying or crazy.}
\wisdom[]{If you're struggling to understand someone's behavior or motivation, understand that it's almost always because of money, fear, or both.}
\wisdom[]{If you have cool stickers, use them. Put them on things. Be carelessly joyful about using your stickers. If you die with a collection of dozens of cool stickers that you never used, you did it wrong.}
\wisdom[Related]{food is for eating, heirlooms are for using, champagne is for drinking, and fancy clothes are for wearing. You are not a fucking docent, and the Pope is not coming to your house.}
\wisdom[]{Whenever you meet someone new, ask them what they're most excited about right now. Everyone interesting is excited about something right now, and they'd probably love to tell you about it.}
\wisdom[Related]{When you meet a child, ask them if they would share the coolest thing that's happened to them this week. You can also ask them about their favorite food. Kids love food and have many thoughts about it.}
\wisdom[]{Whenever you're considering escalating any relationship, ask yourself whether you'd be okay with getting ten times more of them. In other words, consider whether a lot more of "how they are" is a thing you really want to pursue.}
\wisdom[]{Take a walk in a place that has lots of leaves and grass and other natural, irregular patterns. It stimulates dopamine, plus you should probably be walking more anyway. (Thanks, Tom L.)<!-- - Related. This is why the carpeting in casinos and hotels has such whackadoo patterns. They're squirting your brain parts with free happy juice. From a scientific standpoint.}
\wisdom[]{Once your party has been seated, always order a large pepperoni pizza *for the table*. (Thanks, John R.)}
\wisdom[]{Treat every person you encounter as though they are having a way worse day than you.}
\wisdom[Related]{ask yourself how you might become the least annoying stranger that a given person met today. If you became the subject of a private anecdote, how great would you feel about hearing it?}
\wisdom[]{If you're not sure what you want, it's almost definitely more sleep.}
\wisdom[]{Your kids are not little versions of you; they are little versions of themselves. So, don't be sad or alarmed whenever they are becoming something different from you. Because, they will become *lots* of things that are different from you, and that's arguably the whole point. It is inarguably a thing that you need to cheerfully celebrate and support.}
\dashdash
\wisdom[]{If you have trouble keeping up with washing dishes, cutlery, or cooking stuff, you may have too much of them.}
\wisdom[]{Be mindful about giving gifts. A gift you give with *any* expectation is a burden, and people rarely enjoy being given a burden.}
\wisdom[]{In thinking about optimizing how you work, try to distinguish between the parts of your job that are *necessarily difficult* versus the parts that are harder than they actually need to be. The former is the reason that you get the big bucks, and the latter is the reason why you may often feel like the bucks should be bigger.}
\wisdom[]{Be gracious when someone points out a dumb error that you made. Especially when it comes from someone whom you respect. They're doing it because *they like you*, and because they pay attention to stuff that you do.}
\wisdom[Related]{to bad faith actors, you may wish to say something like, "You might just be right" or, "Yeah, life sure is pretty complicated." Answers to bad faith can and should be extremely personal.}
\wisdom[]{Use shoe trees. It'll make you feel like a fancy duchess, plus it'll make your shoes last a lot longer.}
\wisdom[]{Bring along an extra pen that you like.}
\wisdom[]{Buy supplies before you need them and gadgets *after* you need them.}
\wisdom[]{Three is two, two is one, and one is none.}
\wisdom[Related]{change the toilet paper *before* the roll runs out. There is no reward for using the last slice, and, also, you are not seven.}
\wisdom[]{Try to always have fresh lemons around.}
\wisdom[]{Write down the travel items that you forgot to pack *while you're still traveling*. You'll never remember things you've forgotten once you're back home. Choose to lean into your annoyance with yourself.}
\wisdom[]{Order more appetizers and fewer daily specials.}
\dashdash
\wisdom[]{If you want something in life, consider just asking for it. Your friends, clients, and romantic partners are probably not mind-readers. (Thanks, Harry F. H. M.)}
\wisdom[Related]{if you want a sex thing that you think is a little weird—and your partner is a healthy adult—just tell them. They probably want a weird sex thing too, and if you could just give a special thing to each other, how cool would that be?}
\wisdom[]{Be circumspect about which strangers are allowed to alter your mood.}
\wisdom[]{Get new socks.}
\wisdom[]{*Collecting* and *Hoarding* are insanely different things. It's only a "collection" if it's actively displayed and curated. Six unread magazines, on the other hand, is not a collection; it's a recycling project and a personal affliction.}
\wisdom[]{Practice being on vacation before the vacation officially *officially* begins. Start by treating every day off like an actual vacation day. You are unplugged and unreachable, and I empower you to disappear into fun.}
\wisdom[]{Change the soap in your shower way more often.}
\wisdom[]{Listen to a record you liked when you were fifteen.}
\wisdom[]{Take more photos and videos that include the faces of people you love.}
\wisdom[]{Write at least a paragraph a day. Of *something*.}
\wisdom[]{Throw out all shitty scissors.}
\wisdom[]{Bring in your neighbor's trash cans.}
\wisdom[]{Talk to your pets, and remind them they're not so bad—considering.}
\wisdom[]{Close the door behind you.}
\wisdom[]{Except: Always hold the door.}
\wisdom[]{Say "thank you." And mean it.}
\wisdom[]{Try to fix more stuff than you break.}
\wisdom[]{Calm down. (Thanks, Alex C.)}
\dashdash
\wisdom[]{Everybody is doing the best they can each day. Even though what they can do is rarely enough.}
\wisdom[]{To entertain a child, it helps to know which things delight them and which things terrify them.}
\wisdom[Related]{most kids can be surprisingly entertained by your making them just a little bit terrified.}
\wisdom[]{Pay attention to the times of day when you tend to have the most and the least energy. Schedule your future days accordingly.}
\wisdom[]{You have no more than about 40 contiguous hours in which to "catch up" on sleep. Viz. you can't use Saturday morning to refill on sleep you missed last Monday night.}
\wisdom[Related]{you definitely need more sleep.}
\wisdom[]{Maybe almost never say anything about how someone looks ever.}
\wisdom[Related]{if you *are* commenting on how someone looks, only ever compliment them on a thing that they have chosen.}
\wisdom[Relatedly related]{but, yeah, maybe still almost never say anything about how someone looks ever.}
\wisdom[]{Whenever someone insufferable keeps finding a way to mention specific large amounts of money they've spent, you may wish to respond by saying, "Wow! That is a ***lot*** of money!" You may also wish to keep saying this until they leave. (Thanks, Chris C.)}
\wisdom[]{If you feel uncomfortable in small talk situations, first ensure you're doing a *really* good job of listening. No, like: actually listening instead of thinking about how uncomfortable you are about being in the conversation.}
\wisdom[Related]{great listeners ask good questions. "And what year was that?" and "Wow, did that feel really weird at the time?" and "Yikes, was that as terrible as it sounds?" are the sorts of things humans ask one another when they're actively listening to what someone is saying.}
\wisdom[]{Before you move out of a house or an apartment, consider hiding a friendly note to the future tenants. Future tenants love finding friendly hidden notes.}
\dashdash
\wisdom[]{Summon a memory you dislike, and then consider how you might feel different about it tomorrow if you weren't ashamed about it. Then, consider not feeling ashamed about it.}
\wisdom[]{Learn about Chesterton's Fence. Then, actively resist altering a given situation before you understand the reasons why it's remained unchanged for so long. (Thanks, G. K. C.)}
\wisdom[Related]{always read the room. When entering any new situation, be practically invisible and absolutely non-assertive until you can gauge what happened before you arrived—and how it likely felt.}
\wisdom[Relatedly related]{try to avoid beginning a sentence with "Why don’t you just…?" It often indicates that you have approximately zero relevant experience with what has made a given problem such a problem.}
\wisdom[]{Generally avoid clothing that's more interesting than you are.}
\wisdom[]{Any noun can be made a lot funnier by placing the word "prescription" in front of it.}
\wisdom[]{Try not to guess too much about other people's motivations. Yes: even when you can easily imagine what they are guessing about yours.}
\wisdom[]{Clean your eyeglasses at least as often as you defecate. Whether you choose to conduct these activities at the same time is not my concern.}
\wisdom[Related]{buy some decent toilet paper. You're a fucking adult.}
\wisdom[]{Whenever you and an acquaintance find yourselves at an impasse, consider that you are probably starring in different movies. Then, ask yourself how you might be looking in *their* movie.}
\wisdom[]{If you feel like you have to ask a second person to smell the milk, just throw it away.}
\wisdom[]{If you can afford the dinner, you can afford the tip.}
\wisdom[]{If you have time to check email, you have time to *do* something about it.}
\wisdom[]{There are no "bad words." Apart from "moist, "succulent," and "craveable."}
\wisdom[]{No one has ever died wishing they'd spent more time documenting their "minimalist desk."}
\wisdom[]{Negative feelings are like cockroaches. You can chase them away for a few minutes, but they'll always come back as soon as you look someplace else.}
\wisdom[Related]{being chased makes feelings a lot stronger.}
\wisdom[]{In life, you *are* your options. (Thanks, Dan I.)}
\dashdash
\wisdom[]{Almost always order the specific food that is mentioned in the name of the restaurant. If you go to "Sally's Sirloin Shack" and order the sushi sampler, you're probably not hooked up right.}
\wisdom[]{Give kids the opportunity to learn and practice new things in a low-stakes environment. Failure is important in life, but it needn't always be costly or dangerous.}
\wisdom[Related]{also do this for everyone else. Including yourself.}
\wisdom[]{People often unintentionally screw up because they don't realize they're thinking more about their next thing than their current thing. In life, your next thing will tend to go way better if you focus first on not screwing up your current thing.}
\wisdom[Related]{build a grocery list *before* you go to the store. Because, once you're *at* the grocery store, you're mainly thinking about getting out of the grocery store. Rather than, say, two-ply toilet paper.}
\wisdom[]{Your calendar represents a portfolio of promises to your future self. Treat it that way.}
\wisdom[]{Thus: The only events allowed on a serious person's calendar are commitments about time, location, and effort that will die if they are not successfully completed on a specific day. Full stop.}
\wisdom[Corollary]{If you're only tentatively committed to a calendar item—especially if the time of the event has not been mutually confirmed—title the event using Spanish-language questions marks. A future event like "`¿Pick apples with Aunt Sue?`" successfully blocks out the time while also affording a quickly scannable reminder about events that still need to either be formalized or deleted.}
\wisdom[]{Security is a continuum that requires context, updates, and almost infinite trade-offs. Latching your screen door will not prevent a determined person from entering your home, but it will help keep the honest people honest. (Thanks, Chris C.)}
\wisdom[]{There are things that you always want with you when you leave the house. Keys, wallet, phone, etc. That's why, when you *are* at home, they always (and only) ever belong in exactly one place—a bowl, a shelf, a hook, or what have you. Setting them down anyplace else is madness, and you know in your heart that this is true.}
\dashdash
\wisdom[]{Thoughts and feelings are real, but they do not have to define you. Remember that you are the sky—*not the weather*. (Thanks, Pema C.)}
\wisdom[]{If you're going to a party, always bring a bag of ice. The host will  appreciate it, because nobody has ever been annoyed about receiving something useful that just turns into water once it's no longer useful.}
\wisdom[Related]{remember whose event this is. If it's not your party, you don't get to pick the music, the guest list, or the vibe. If you can't find a way to get along, just leave.}
\wisdom[Relatedly related]{a party is only as good as the people who attend it. Especially you. Be helpful, be fun, and delight at least one new stranger.}
\wisdom[]{Rather than curating a collection of well-rounded students, strive instead to attract the constituents of a well-rounded *class*. (Thanks, Rab T.)}
\wisdom[]{Avoid situations that someone you love might later have to explain on a medical or government form.}
\wisdom[]{Think of every email you send as a pebble. To you, it may seem like a comically small thing—almost non-existent in size and weight. But, to a recipient already holding hundreds of other people's pebbles, receiving even one more tiny pebble is not without a cost.}
\wisdom[]{Until you learn the distinction between an adversary and an enemy, you're likely to accumulate quite a few of the latter.}
\wisdom[]{Always speak as though quoting *Fight Club* costs at least forty dollars.}
\wisdom[]{Loving books and loving reading are very different things. One is about treasuring the priceless gift of the written word, and the other is about constantly telling strangers how much you love books.}
\wisdom[]{If you can manage it, avoid teeing up a media suggestion with "I couldn't recommend this more." Because if the listener (understandably) zones out on the journey to that crucial last word, it really sounds like you kinda hated it.}
\wisdom[]{Learn the difference between being "busy" versus being *time-constrained*. Time constraint is an immutable fact of adult life; while it can (and should) be managed, time constraint can never be eradicated. Being "busy" should be better understood as a time-limited result of an earlier management error. If you find that your busy-ness has become permanent, you must learn to manage earlier, differently, and much *much* better.}
\wisdom[Related]{start acting like your life matters.}
\wisdom[]{Always eat and shit *before* your flight. These are just two of the numerous human requirements that are best performed anywhere that's not a commercial airline flight.}
\wisdom[]{Start noticing how often you explain away stasis, trauma, codependence, abuse, or generally unwholesome situations because of something you claim that you "**have**" to do.}
\wisdom[Related]{consider accepting that you do not actually *have* to do anything except die.}
\wisdom[]{Make time to write the thank-you note *before* you open the present.}
\wisdom[]{Don't call it a "warning" if it's really a *threat*. "Be careful walking to school" is a warning; "I am going to bomb your school" is a threat.}
\dashdash
\wisdom[]{When you first try to meditate, it'll feel like you're doing it wrong and are terrible at it. Eventually, you'll figure out that being terrible at meditation and feeling like you're doing it wrong is kind of the whole point of meditation. In truth, as long as you keep getting back to your practice, you are actually great at meditation, and you are doing very well at it.}
\wisdom[Related]{you'll rarely earn points in life for repeatedly thinking about something you're not doing. Unless worrying and fretting about something you're not doing makes you happy, you may wish to worry and fret less.}
\wisdom[]{If a given thing you've decided to do goes flawlessly, what's the best possible outcome you can imagine? Keep this question in mind before you, say, jump onto a moving car or scream at a baby.}
\wisdom[]{Don't buy food to get a free toy, and don't buy toys to get free food.}
\wisdom[]{The best optimism is *earned*. Everything else is just magical thinking.}
\wisdom[]{Everybody feels bad sometimes, but try not to feel too bad about it. Choosing to feel bad about feeling bad is ultimately optional.}
\wisdom[]{We don't get to pick what we love or what makes us cry. Seek relationships with people who support the things that make you weepy or horny.}
\wisdom[]{Try to save some parts of your life to be just for you. Including some special things that you're happy about or are even a little proud of. If your only private things are shameful things, you will become very sad and will eventually despise your own company.}
\wisdom[]{Learn to love at least one snack that can be stored in a glove box without going bad.}
\wisdom[]{To delight a busy person, learn to ask questions that can easily be answered with a single word: "Yes."}
\wisdom[]{Whenever your first solution to a problem feels like it should involve buying something plastic at The Container Store, consider a second solution.}
\wisdom[]{Most anecdotes that begin with the word "Apparently…" do not end well.}
\wisdom[]{Open your mail over the recycling bin.}
\wisdom[Related]{put a trash can anyplace your dominant hand repeatedly wants to let go of trash.}
\wisdom[]{If you think you're immune to making unintentional cognitive errors, you should read this sentence over and over for a few minutes.}
\wisdom[]{Before you buy a new book, try to get a copy from the library—or just download an ebook sample. If you can't manage to finish reading the Kindle sample, you certainly oughtn't pay to not read the rest.}
\wisdom[]{Reading Shakespeare is all about finding the right velocity. Read it too fast, and it won't make any sense. Read it too slow, and it'll make even less sense.}
\wisdom[]{If your guitar sounds out of tune, it's almost definitely the B string, the G string, or both.}
\dashdash
\wisdom[]{The people most obsessed with the (supposed) hypocrisy of strangers are often the people who most dread strangers uncovering *their own* hypocrisy. This makes many of these people a little insufferable and not very fun to hang out with.}
\wisdom[Related]{there are numerous things in life worse and more damaging than hypocrisy. Including the implicit belief that eventually changing to become a little less of a jerk must be excoriated for its rank "hypocrisy."}
\wisdom[]{Before deciding that you have solved a problem, it's useful to ask yourself whether you understand what *caused* the problem—as well as knowing precisely how your specific choice of solution has "fixed" it. If you mostly just kept trying various random things until something seemed to improve, you just got lucky. Which is different.}
\wisdom[]{Whenever you acquire anything that has a manual, Google for a PDF of it, then drop it in a folder on your cloud service of choice.}
\wisdom[]{It's mostly to young people's credit and benefit that they rarely realize that youth and its advantages are anomalous. Youth is all you have ever known in life until you reach an age when you are no longer young. At which point, you are likely to develop a strong hunch that your own youth was absolutely an anomaly.}
\wisdom[]{Unless you are cannily identifying a specific logical fallacy, the question has likely been *raised*. Not begged.}
\wisdom[]{Consider using less corn starch than the recipe demands.}
\wisdom[]{Being good at arguing is not the same thing as being right; being bad at arguing is not the same thing as being wrong.}
\wisdom[]{It is normal and human to fear things that you don't understand. But, seeking to understand something does *not* mean you have to like it or agree with it. It may, however, suddenly make some of your fears feel refreshingly optional.}
\wisdom[]{Carry more Imodium than you think you will ever need. They are compact in size, light in weight, and miraculously effective at keeping you from suddenly needing to shit egregiously at times and locations when you'd prefer not to be shitting egregiously.}
\wisdom[Related]{Imodium works by sucking liquid out of your bowels, so compensate by drinking more water than you normally would. The Imodium will still work fine.}
\wisdom[Relatedly related]{several Imodiums will fit conveniently in the right-side "watch pocket" of a pair Levi's. Ditto guitar picks, a small USB drive, or some quarters with which to play the excellent 1981 arcade game, "Galaga."}
\dashdash
\wisdom[]{Most parents understand it's their job to keep their child from dying. But, as the kid gets older, it also becomes increasingly vital not to prevent them from living. This is very difficult.}
\wisdom[]{If you're going to participate in a dumb fight online, do it in the late afternoon. That way, you're less likely to blow a perfectly good working day being all mad. Plus, you'll both tire yourselves out eventually, and maybe you'll do better tomorrow.}
\wisdom[Related]{do not participate in dumb fights online.}
\wisdom[]{Minimize the number of conversations you have through a closed bathroom door. Unless you're outside the door and there's a fire, or you're *inside* the door and you're out of toilet paper. Otherwise, have a little dignity, and wait for the door to open.}
\wisdom[]{Don't be too thirsty in your quest for gratitude and acknowledgement. If you want to help someone and have the requisite skills, *just help them*. A person with a problem who also has to graciously manage needy communications about that problem now very much has two problems.}
\wisdom[Related]{if you *really* want to help someone, offer something extremely specific. "I'm here for you! 😬👍" is not nearly as cool as "Can I drop off a lasagna at 4?"}
\wisdom[]{You cannot alter history by choosing not to use words. But, you do stand a small chance of improving the future by choosing better words.}
\wisdom[]{Learn to play guitar. It takes, like, two weeks. And, once you kinda know what you're doing, you can play songs you like or just keep your hands occupied while you're thinking.}
\wisdom[Related]{a ukulele-sized six-string guitar (or, *guitalele*) is a portable, affordable, small-footprint instrument that you can just have around the house. Plus, they're super fun to play.}
\wisdom[]{While it's weird to *invent* a tradition, start noticing the things that have made you happy when (or because) they've happened more than once. Then, consider acknowledging those things as a tradition.}
\wisdom[Related]{some of the best traditions are silly or weird or trivial traditions. Traditions related to repeating events like holidays can be a lot of fun and even very moving. But, you may discover that your new favorite traditions are just the dumb group rituals that arise amongst people who like each other and quietly treasure the hours and years they've spent sharing mundane things.}
\wisdom[]{Xerox or scan the non-currency contents of your wallet twice a year. You'll likely regard this as a weird waste of effort until the day it saves your bacon.}
\wisdom[]{Tip the hotel housekeeping staff. Tip them handsomely. If you had to clean *their* toilet, wouldn't you appreciate a little something for the effort?}
\wisdom[]{Be conservative about how much stuff you accumulate because you imagine someone "might want it someday." If you're warehousing something that feels like a minor heirloom, ask the person if they want it right now, and when they inevitably say, "Oh, my God, ***NO!***" be gracious in defeat, and just find it a good home with a charity or a trash can.}
\wisdom[Related]{charities, homeless shelters, and schools do not need your filthy or broken shit; they need your money. So, just give them some cash, and stop treating "worthy causes" like a guilt-free DMZ for your junk.}
\wisdom[Relatedly related]{quit buying products because "a portion of the proceeds goes to charity." If you actually care about a cause, give that cause some actual money. Then, you'll know *all* of the proceeds have gone to charity.}
\wisdom[]{If there's a book that means a lot to you, buy five print copies. It helps the author, plus now you can give a free copy to a friend whom you think would love it.}
\wisdom[]{When you get coins as change, throw them in a jar and forget about it.  That's now your baby steps toward savings, a vacation, or what have you.}
\wisdom[Alternatively]{leave your spare coins in a tidy little stack right outside the store you just exited. I'll bet you a chicken dinner the person who finds them could use them a lot more than you.}
\wisdom[]{When you die, your family will be charged \$100 for every time you've ever honked your car horn. I cannot tell you how I know this, but please just understand with all sober certainty how very important it is that you never again honk your car horn.}
\dashdash
\wisdom[]{If you're noticing a new name or phrase being used to refer to something you know by an older or more familiar name, try to keep an open mind. Rather than focusing on your own annoyance or discomfort, ask yourself in which direction the new name is encouraging people to move. If it feels like a good faith step toward inclusion, authenticity, kindness, or greater humanity, maybe set aside your priors and make the effort to take those positive steps right along with it.}
\wisdom[]{But: remain skeptical about quickly adopting new jargon from the worlds of business, technology, or journalism. Especially if it's that sort of new jargon that enthusiastically trades clarity and precision for deliberate opacity or cheap novelty. The track record for jargon's longterm contributions to society has been bleak.}
\wisdom[]{Be suspicious of people who like being owed a favor. Especially if it's not a favor that you requested. Often, these people are eldritch monsters who thrive on accumulating goodwill for darkly selfish reasons.}
\wisdom[]{If the soap in a guest bathroom is new and shaped like anything besides a bar of soap, do not use it. Also, do not eat it. Because I know you kinda want to. Especially those shiny little sea shells.}
\wisdom[]{If you're not sure who's doing your emotional labor, it's probably everyone you know.}
\wisdom[]{When you order delivery food from a new place that looks promising, use it as an opportunity to explore. Set a baseline by ordering a dish you love, especially if it seems hard to screw up. But, as a flyer, also consider ordering an appetizer or side dish they claim people love.}
\wisdom[]{When you order delivery food on behalf of a child, and you have a special request whose contravention risks ruining their meal, don't be squeamish about it. Just say, "No mustard, please. It's for a kid."}
\wisdom[Related]{maybe think twice before telling the order-taker about a (non-existent) allergy that you claim gravely constrains your diet. A lot of people do this, restaurants know it, and it ends up harming the people who actually do have an actual allergy. A strong preference rarely rises to the level of being an emergent medical issue.}
\wisdom[Relatedly related]{if you really don't want mustard on your food? Yeah, maybe just tell them "It's for a kid."}
\wisdom[]{In between "yes" and "no" is a powerful thing called *the qualified 'yes.'* Responsible and self-aware adults have every right to place conditions on their agreeing to do almost anything. And, trust me that the people who either reject or ignore this seeming subtlety are rarely the sort of people to whom you want to be making *any* kind of commitments.}
\wisdom[]{If someone asks you to critique their work, try to gauge what it is they're actually looking for. Many people just want a friend's praise, others may be seeking insight on how close they are to being finished, while a rare few crave the most candid and withering feedback you can muster.}
\wisdom[Related]{people who are *accustomed* to receiving honest feedback are often really good at providing focus on what they specifically need help with, as well as context for why your advice would be useful to them right now. If their big question is "So, do you love it?" you are likely not talking to one of those people.}
\wisdom[]{Success in relationships—especially in marriage—will largely come down to how many things only one person is ever allowed to be right about.}
\dashdash
\wisdom[]{Piles can be a quick and dirty way to tidy an area or organize a bunch of stuff. But, temporary piles tend to devolve into permanent piles, and permanent piles are struggling to warn that you are becoming neither tidy nor organized.}
\wisdom[]{When removing a full bag from a trash can, consider dropping a folded-up future replacement bag at the bottom of the can. This will seem weird and unnecessary until the day you hadn't realized you've run out of trash bags, and then—hey—free bonus bags.}
\wisdom[]{IKEA's blue *FRAKTA* shopping bags are one of life's low-key power tools. They fold ridiculously small, so they can live unobtrusively in places like automobiles or even other bags, like suitcases. Because, some day you may suddenly wish you had *freaking nineteen gallons* of extra bag with you.}
\wisdom[]{If you're going to hide a spare house key, consider putting it somewhere that's not right outside your house. Even an industrious prowler is less likely to gain easy access to your home if your unmarked spare is tucked behind a neighbor's picnic table or buried under a rock in the park.}
\wisdom[]{The potent powdered stuff used to clean espresso machines is also a kick-ass solution for soaking nasty cookware.}
\wisdom[]{You are using too much dish soap. Use less dish soap but hotter water.}
\wisdom[]{Buy some manila file folder jackets, and deploy them for any occasion or destination where random paper kipple is likely to pile up. You know that wad of receipts and boarding passes and theme park maps and whatnot that accumulate every time you travel? Anticipate the wad, and then easily manage it with your new folder friend. It would like to live right next to your room's ad hoc charging station, please.}
\wisdom[]{When you get multiple hotel key cards for the multiple people in your party, use stickers or a Sharpie to identify whose card is whose. Seems dumb until you realize too late that one person has left the room with three cards and two people have left the room with zero cards.}
\wisdom[]{If you're about to depart for an event that requires physical pre-purchased tickets, every person in your party has to hold their own ticket to their own forehead. Then, you don't leave until the entire group agrees that they've all seen each person's ticket. (Thanks, Phil S.)}
\wisdom[]{You're either on the bus or off the bus. (thanks Ken K.)}
\wisdom[]{related: this also goes for doors, lids, locks, and relationships.}
\wisdom[]{relatedly related: your ear buds are only ever allowed to be in one of two places *ever*. They are either in your ears or in their charging case. That is it. Ever. (Thanks, John S.)}
\dashdash
\wisdom[]{Whenever you switch to having two of something, change is happening, and you need to be aware of that. If you had one lover, but then suddenly find yourself with two, change is afoot. If you recently had one-thousand dollars, but are now down to two? Something big has changed.}
\wisdom[Related]{this also goes for keys, college majors, addictions, indictments, toilet paper rolls, wishes from a genie, and almost everything else. "Two" often means shit is on the precipice of getting extremely weird.}
\wisdom[]{Like it or not, you're always practicing *something*. Put another way, whichever muscle you exercise the most can't help but strengthen. Often to the detriment of others.}
\wisdom[Related]{okay, that was mostly an analogy. What I'm saying is that you will eventually *become* whatever you frequently do. So, be picky about frequently doing only the things that help aid whatever it is you *want* to become.}
\wisdom[]{No one speaks through a translator; it is we who *listen* through a translator.}
\wisdom[]{Your refrigerator is not a library or a hope chest. So, if you decide to save some leftovers, write the current day of the week on them. Then, when you rediscover your treat 3-5 weeks from now and wonder "Now, *which* Sunday was that?" Yeah. Time to deeply curate your odd little food museum.}
\wisdom[Related]{good intentions to reduce waste are best operationalized before acquiring the things that tend to get wasted. Live the rule of the buffet: "Take what you want, but eat what you take."}
\wisdom[]{Life would be simpler if we got to choose which things made us a little less dumb or ugly. If something has made you rethink how you roll, just humbly take the win, and keep getting better without regard to what caused it and when.}
\wisdom[]{As you get older, you will increasingly fear losing power, and you will become bitter, defensive, and angry about change. Curiosity, acceptance, and exposure to new people can help with this. But, man are you ever going to get weird about people with purple hair who are not afraid of you.}
\wisdom[Related]{almost no one has ever *actually* been afraid of you.}
\wisdom[Relatedly related]{the only people who were ever *actually* afraid of you were the handful of people who loved you and desperately wanted you to love them back.}
\wisdom[]{For chrissakes, get a new kitchen sponge already.}
\dashdash
\wisdom[]{Remember that, like babies and balls, you can bounce. The extent to which any given event—often an imagined event—might derail or even destroy you is, at least in small ways, still something that's in your control. Especially when you remember that you can bounce.}
\wisdom[]{Whenever the start time for a call or similar gets delayed—even for a theoretically specific time in just a few minutes—consider asking your tardy colleague for a "*bump*." "Text me when you're ready—but, then give me 15 minutes" allows your brain to move fully out of "waiting mode" and into "doing stuff mode." If the event ends up starting promptly at the new time? (Which it won't.) Great. And, if it doesn't start on time? (Which it won't). No problem. You will have disappeared into an appropriately-sized task instead of being all mad about transitively wasting time and attention that could have been more skillfully utilized.}
\wisdom[]{In considering whether and how often to ingest a source of new or updating information, always try to ask yourself: "What might this make me want to see, think, or do differently—and how might that be beneficial?" Whenever you estimate that your honest answers should be "Nothing important" and "It won't," be cautious about ingesting it at all. I promise, you do not actually need to know everything all the time.}
\wisdom[]{One easy and quick free writing exercise is to pick a word you just heard that tickles your brain. For three minutes and without pausing, write as many associations as you can think of with that word. Weirdly, something will grab you and stimulate your juices, and you may find yourself writing for longer than three minutes.}
\wisdom[Related]{try to avoid saying that you're going to "make yourself" do something. Instead, take a breath, and describe what you hope to *find yourself* doing. Inhabit your intentionality, and quit trying to knock down a fully unlocked door.}
\wisdom[Relatedly related]{a *goal* can be a fine thing that describes a desired outcome defined by a positive change that you desire. On the other hand, think of an *intentionality* as the mindful way you'll choose to conduct and focus yourself on the way to a "goal." Or pretty much anything else for that matter.}
\wisdom[]{Before telling someone else "how they are" or "what they should do," ask yourself how much those very important notes may actually be all about *your* shit and absolutely zero about theirs.}
\wisdom[]{Give each of your houseplants a funny name. Then, *use* it. If you have one plant and have no ideas, you may choose to call your friend, "Planty." That's what I do.}
\dashdash
\wisdom[]{Sometimes, when you're about to ask someone for help, you have a hunch that it might be something you could probably figure out on your own. To test your hunch, *draft* (but do not send) your question as the smartest and most succinct message you can manage. Yes, reading it back to yourself may end up solving your actual problem. But, it nearly always helps to clarify whether it's something you actually *could* figure out on your own.}
\wisdom[Related]{knowing whether a problem is something you might be able to figure out on your own is arguably as important as actually "solving" the problem. Just ask anyone who's ever assumed they knew how to fix their own brakes or diagnose a weird lump.}
\wisdom[]{If you cannot understand why someone was traumatized by an event, maybe ask yourself who that says more about.}
\wisdom[]{The second step in almost any project will involve the mindful introduction of *infrastructure*. I don't even need to know what your first step is; I can almost guarantee that there will be no third or fourth step if you aren't canny about getting the proper initial scaffolding in place.}
\wisdom[Related]{a lack of infrastructure is also why nearly all New Year's resolutions fail. Human will is a lot like a new locomotive; it's basically useless without the rails, controls, and signals that make the right thing the easy thing.}
\wisdom[]{Always act as though you're going to survive. If you don't survive, it probably won't matter how you acted. And, if you do survive (which you probably will), I'll bet it's at least partly because you *acted* like you were going to survive. Try it.}
\wisdom[]{"Creative work" and "independent work" are rarely the same thing. In practice, a great deal of creative work is contracted or controlled by other people, and a striking amount of independent work is just shockingly non-creative. This is not a bad thing, but knowing about it in advance can save you a ton of unnecessary disappointment.}
\wisdom[]{Unless you *really* know what you’re doing, never give anyone underwear as a gift.}
\dashdash
\wisdom[]{Kids are unintentionally curious. What happens when I flush this rug? Would my cat enjoy wearing a fancy dress? Could this LEGO head also be a cool hat? Without regard to topic, try not to lose what’s left of your unintentional curiosity.}
\wisdom[]{Some people like to tell you what to pay attention to. Many want to determine how you should feel about it. Others specialize in advising exactly what you should do as a result. Just be aware that these are each different things, and it's probably wise to avoid people who are adamant about demanding to control all three.}
\wisdom[Related]{being good at one of these things rarely means someone's also great at the others.}
\wisdom[Unrelated]{this also goes for the notional expertise portfolio of billionaires. Sometimes, perversely so.}
\wisdom[]{In the world of independently making stuff, success with an audience can be thought of as a slice of cake that is impossibly slim but improbably deep. Stay focused on delighting the right tiny wedge of the best baked good you can make and there's potentially no limit to how tall it can get.}
\wisdom[Related]{please do not eat your fans. They are not actually cake.}
\wisdom[]{Great cities often thrive on the number of options that ordinary people have for getting from Point A to Point B—and for potentially exploring unfamiliar terrain both between and beyond those points. Limit those options to less than a handful, and both your city and its points start to lose a lot of their appeal. (Even if you have a *really* nice car.)}
\wisdom[]{Treat recipes like a journey, not a destination. Once you feel like you're mostly nailing the basic instructions, start looking for ways to make it your own. Remember, you are probably not being scored.}
\wisdom[]{Imagine standing on a footbridge and looking down as a fast, choppy stream rushes beneath you. Keep watching the stream—but try not to jump in the water. *There*. You just meditated a little.}
\wisdom[Related]{if it's useful, you might want to think about meditation as regular office hours with yourself.}
\wisdom[]{When you're in your 20s, be advisable about accumulating a lot of long-term obligations. If, when, or as your options start to diminish, the sunk costs of theoretical security often become a costly millstone.}
\wisdom[]{First, make a hole. Because before you can properly clean, organize, or even *purge* a given area, you'll need to know where the stuff that's not currently where you want it to be will go. Yes, especially the trash.}
\wisdom[]{If someone else is doing your dishes, be guarded in how often and loudly your criticize how they do it.}
\wisdom[Related]{This also goes for almost everything else.}
\wisdom[]{You may not always get to choose what makes you emotional, but you have more agency than you might think about what makes you *irrational*. Which is really kind of the point when you think about it. Which you should.}
\wisdom[]{Some people are skeptical about the teaching of any topic or thing that they don't personally already know and understand—or more often, something that they *think* they know and understand.}
\wisdom[Related]{You can draw your own conclusions as to what this reveals about those folks' intellectual curiosity. But do understand: the less they know, the less they think *you* should know.}
\dashdash
\wisdom[]{Find moments of grace where you can feel gratitude for the gift of having a small place in a world that is much bigger, older, and more interesting than you are.}
\wisdom[Related]{one excellent opportunity to practice this is every time you smile and raise your glass to join a toast.}
\wisdom[]{There are Finish-the-Can people and Not-Finish-the-Can people. You know in your heart which kind you are. As do the people with whom you live.}
\wisdom[]{If you think about someone when they're not around and you find yourself wondering whether there are ways you could make their life a little better, consider that you might love them.}
\wisdom[]{Hats are like PowerPoint decks. Probably best to just steer clear until you understand why most men are terrible at it.}
\wisdom[]{It can be frustrating when someone won't pay attention to the thing you wish they'd be paying more attention to. Consider that other people may sometimes feel that same frustration about you.}
\wisdom[]{Don't say you've lost something if you've merely *mislaid* it. "Lost" is a really special word, best reserved for describing a precious thing you've accepted you'll never see again.}
\wisdom[Related]{if you *have* mislaid something, it's probably either underneath or inside of something else.}
\wisdom[Relatedly related]{remember to hunt for mislaid items in places where they aren't supposed to be. If a thing was where it's *supposed* to be, then you wouldn't still be looking for it, would you?}
\wisdom[]{If I were to pitch somebody on my favorite bit I’ve picked up from mindfulness, I’d suggest starting to frequently notice how you feel about how you feel.}
\wisdom[Related]{if you don't think you sometimes have *big* feelings about how you feel, please permit me to gently recommend your looking into mindfulness.}
\wisdom[]{Get a thin, lightweight rain jacket. Roll it up into a tight and tiny little burrito, secure it with a couple silicone/rubber bands, and just leave it in your backpack. Because no matter how much it's not raining right now, chances are it will rain again someday.}
\wisdom[Unrelated]{responsible project management is the art of calmly preparing for things that aren't happening right now. A manic dash to survive something that's already happening (and which you probably should have anticipated) is not generally regarded as a skillful application of the art.}
\wisdom[Relatedly unrelated]{the use of a rain umbrella should require roughly the same formal testing and certification as an automobile license. If you are in public and using an umbrella irresponsibly, you should lose your privileges. It is not your personal weather sword, and it makes you just *so* much broader and pointier than you realize. Maybe just get a rain jacket.}
\wisdom[]{If you can't be a good example, at least try to become an interesting cautionary tale.}
\dashdash
\wisdom[]{When you need to scoop some ice out of a thing, never use something that could break—especially anything made of glass. Watching someone drunkenly plow a highball glass into an ice bin makes the retired busboy in me wildly uncomfortable.}
\wisdom[]{Occasionally silently remind yourself, "Remember you said you wanted this."}
\wisdom[]{Avoid telling people they're being scared wrong. Demanding that someone become more or less terrified about something is a dick move, and it usually says more about your hangups than their judgement.}
\wisdom[Related]{start noticing the people in your world who try to motivate you with fear. It's often because they don't have anything more appealing to offer. And, they know it.}
\wisdom[Relatedly related]{whenever you can  manage it, don't internalize fear until you've exhausted curiosity.}
\wisdom[]{Anxiety has a way of encouraging you to obsess about a horrible thing your brain just made up. Next time this happens, try to relax and instead just gently acknowledge that sometimes your brain makes up horrible things. Believing these things and obsessing about them is ultimately optional.}
\wisdom[]{You'll almost always eventually regret having named something you made with a pun.}
\wisdom[]{Lessons come where you find them. Not necessarily where you looked for them.}
\wisdom[]{Try to become someone who's fun to teach things to.}
\dashdash
\wisdom[]{Sometimes, your brain decides it’s time to play tennis. So, it starts serving ideas and images at you, each of which seems to require immediate, urgent, and committed volleys in return. Just remember: *it’s your damned head*, and it’s fine if you just want to go sit in the bleachers and watch your brain fire heaters at no one until it tires itself out.}
\wisdom[]{Whenever you notice something, consider saying "*thank you*" in your head. Even or especially if it's something small or random. You're saying thank you to the world for still being there—and thank you to yourself for noticing it.}
\wisdom[Related]{if this strikes you as corny or emotionally disordered, then you, my friend, are not noticing enough things. Which also means you're definitely not saying "thank you" enough. So, honestly, who's the real weirdo here?}
\wisdom[Relatedly related]{*Thank you*.}
\wisdom[]{Never tell someone they look like someone else. If you're in doubt, recall five times you've done this, it was welcomed, and it all turned out great for everyone. You either sound like a sociopath, a person who puts in women in wells, or, likely, both.}
\wisdom[]{All photos are pictures, but not all pictures are photos. If you're talking about a photograph, and you call it a "picture," you sound a little like someone who drinks from a jug.}
\wisdom[]{Sometimes, the problem is not that someone lacked the technology to make what they wanted. Often, they had more than enough technology to make what they wanted, but—yikes—just look at…*what they wanted*. (Thanks, Phantom Editor)}
\wisdom[]{At least consider the option of not having an opinion.}
\wisdom[]{Every time you're tempted to harass your kid about "screen time," mentally imagine that you're actually ranting about, say, "*rectangle time*." Because it makes about as much sense right now. This is not the 12th century, you're not a particularly sagacious elder, and your kid probably doesn't benefit overmuch from adopting your rustic folklore about how concerning machines are.}
\wisdom[]{It's okay to plan a meal around dessert. You're an adult, and people who try to tell you how to eat should be offered dietary advice regarding eating a butt.}
\dashdash
\wisdom[]{If you want to know what someone really thinks, ask them what they think other people really think. Then, ask how they know what other people really think.}
\wisdom[]{Every few minutes, try to do something skillful.}
\wisdom[]{Once you learn something, you can always relearn it. Unless it's something you only *think* you learned. You can't really relearn something you only think you learned.}
\wisdom[Related]{Minimize relationships with the sort of person you've learned you really shouldn't have a relationship with.}
\dashdash
\wisdom[]{Avoid saying things you don’t want to be judged about. You’re already being judged for a shit-ton of things you don’t control, so be cautious about deliberately generating new material.}
\wisdom[]{Most of the easy problems have been solved. If a problem still exists, it's probably because it's either a *really* hard problem, or it's not actually a problem.}
\wisdom[]{Never brag about your weather. You had no agency in creating it and deserve no credit for living in it.}
\wisdom[]{Lean away from self-pity. Self-pity is also self-basting. Which is gross.}
\wisdom[]{When you get hit with a sock full of coins, it doesn’t really matter that much whether they're pennies or quarters.}
\dashdash
\wisdom[]{Life is a tile puzzle. You're often going to need to move some around a bunch of stuff that seems completely unrelated to your theoretical goal. At least at first it will seem completely unrelated.}
\wisdom[]{Avoid using your hand as a tool. Almost always prefer to use your hand to *manipulate* an appropriate tool. You're going to need your hand for other things.}
\wisdom[]{Nearly everything that happens in the world doesn't involve you. How often do you think about this?}
\wisdom[]{Your photos will get more interesting once you've learned to lie prone on the ground for a surprisingly long time.}
\wisdom[]{Sometimes we make things to teach, but often we make things to learn.}
\wisdom[]{When you’re trying to help someone, focus on the help that the person needs rather than the help you feel like giving them.}
\wisdom[]{At least twice a day, focus on doing something very skillfully.}
\wisdom[]{It's not laws that protect you; it's people observing laws that protects you.}
\dashdash
\dashdash

\end{document}  
