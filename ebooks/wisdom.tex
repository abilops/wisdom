% XeLaTeX can use any Mac OS X font. See the setromanfont command below.
% Input to XeLaTeX is full Unicode, so Unicode characters can be typed directly into the source.

% The next lines tell TeXShop to typeset with xelatex, and to open and save the source with Unicode encoding.

%!TEX TS-program = xelatex
%!TEX encoding = UTF-8 Unicode

\documentclass[12pt, twosided]{book}
\usepackage{geometry}                % See geometry.pdf to learn the layout options. There are lots.
\geometry{a5paper}                   % ... or a4paper or a5paper or ... 
%\geometry{landscape}                % Activate for for rotated page geometry
%\usepackage[parfill]{parskip}    % Activate to begin paragraphs with an empty line rather than an indent
\usepackage{graphicx}
\usepackage{amssymb}
\usepackage{pdfpages}

% Will Robertson's fontspec.sty can be used to simplify font choices.
% To experiment, open /Applications/Font Book to examine the fonts provided on Mac OS X,
% and change "Hoefler Text" to any of these choices.

\usepackage{fontspec,xltxtra,xunicode}
\defaultfontfeatures{Mapping=tex-text}

%%% GITHUB does not have good fonts, so sticking to defaults
%\setromanfont[Mapping=tex-text, Size=30]{Times New Roman}
%\setsansfont[Scale=MatchLowercase,Mapping=tex-text]{Gill Sans}
%\setmonofont[Scale=MatchLowercase]{Andale Mono}

\title{Wisdom Project}
\author{Merlin Mann}
%\date{}                                           % Activate to display a given date or no date


\begin{document}%
\includepdf{cover.jpg}
%\noindent\makebox[\textwidth]{
%\includegraphics[width=\paperwidth]{cover.jpg}
%}

%\maketitle

% For many users, the previous commands will be enough.
% If you want to directly input Unicode, add an Input Menu or Keyboard to the menu bar 
% using the International Panel in System Preferences.
% Unicode must be typeset using a font containing the appropriate characters.
% Remove the comment signs below for examples.

% \newfontfamily{\A}{Geeza Pro}
% \newfontfamily{\H}[Scale=0.9]{Lucida Grande}
% \newfontfamily{\J}[Scale=0.85]{Osaka}

% Here are some multilingual Unicode fonts: this is Arabic text: {\A السلام عليكم}, this is Hebrew: {\H שלום}, 
% and here's some Japanese: {\J 今日は}.
\newcommand{\dashdash}{$\ast$}
\newcounter{wcounter}
\setcounter{wcounter}{0}
\newcommand{\middleset}[1]{%
%\vspace*{\fill}
#1 %
%\vspace*{\fill}
}
\newcommand{\wisdom}[2][]{%
		% Increment counter only if this is a top
		\ifstrequal{#1}{}{\stepcounter{wcounter}}{}
		% break page only if this is a top
		\ifstrequal{#1}{}{\clearpage}{\\[2em]}
				\ifstrequal{#1}{}{
				\begin{center}
						\textbf{$\bullet$\thewcounter}
				\end{center}
				}{%
						{\Large {#1:}}%
				}
		\middleset{\Large #2}
}
%\newcommand\done[2][\relax]{%
%    \ifx#2\todo\@todofalse\else
%        \ifx#2\Todo\@todofalse\else
%        \PackageWarning{Floating \string\done\space ignored.}%
%\fi\fi

\include{wisdom.md.tex}

\end{document}  
