% XeLaTeX can use any Mac OS X font. See the setromanfont command below.
% Input to XeLaTeX is full Unicode, so Unicode characters can be typed directly into the source.

% The next lines tell TeXShop to typeset with xelatex, and to open and save the source with Unicode encoding.

%!TEX TS-program = xelatex
%!TEX encoding = UTF-8 Unicode

\documentclass[12pt, twosided]{book}
\usepackage{geometry}                % See geometry.pdf to learn the layout options. There are lots.
\geometry{a5paper}                   % ... or a4paper or a5paper or ... 
%\geometry{landscape}                % Activate for for rotated page geometry
%\usepackage[parfill]{parskip}    % Activate to begin paragraphs with an empty line rather than an indent
\usepackage{graphicx}
\usepackage{amssymb}
\usepackage{pdfpages}

% Will Robertson's fontspec.sty can be used to simplify font choices.
% To experiment, open /Applications/Font Book to examine the fonts provided on Mac OS X,
% and change "Hoefler Text" to any of these choices.

\usepackage{fontspec,xltxtra,xunicode}
\defaultfontfeatures{Mapping=tex-text}

%%% GITHUB does not have good fonts, so sticking to defaults
%\setromanfont[Mapping=tex-text, Size=30]{Times New Roman}
%\setsansfont[Scale=MatchLowercase,Mapping=tex-text]{Gill Sans}
%\setmonofont[Scale=MatchLowercase]{Andale Mono}

\title{Wisdom Project}
\author{Merlin Mann}
%\date{}                                           % Activate to display a given date or no date


\begin{document}%
% Cover image!
\includepdf{cover.jpg}
% Reset page numbers
\setcounter{page}{1}
% For the Markdown "----"
\newcommand{\dashdash}{\\[3em]$\ast\ \ast\ \ast$}
% Counter for each line of wisdom
\newcounter{wcounter}
\setcounter{wcounter}{0}
% Vertically center by removing those comments
\newcommand{\middleset}[1]{%
%\vspace*{\fill}
#1 %
%\vspace*{\fill}
}

% Set the wisdom line
\newcommand{\wisdom}[2][]{%
		% Increment counter only if this is a top
		\ifstrequal{#1}{}{\stepcounter{wcounter}}{}
		% break page only if this is a top-level and not corollary or related
		\ifstrequal{#1}{}{\clearpage}{\\[2em]}
				\ifstrequal{#1}{}{
				\begin{center}
						\textbf{$\bullet$\thewcounter}
				\end{center}
				}{%
						\emph{\Large {#1:}}%
				}
		\middleset{\Large #2}
}

% With the definitions done, we can just pour in the wisdoms!
\include{wisdom.md.tex}

\end{document}  
